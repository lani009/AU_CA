\documentclass{article}
\usepackage[utf8]{inputenc}
\usepackage{kotex}  %
\usepackage{setspace}
\usepackage{algorithm}
\usepackage{algorithmic}


\author{관우 안}
\setstretch{1.2}

\begin{document}
\section{Exercise 2}


\subsection{i) (5 point)}
\\
\\
\\
모든 원소의 값이 같은 배열 A에서 pivot을 가장 오른쪽 값으로 정하고 Quicksort를 시행할 경우, pivot과 다른 모든 원소가 동일하기 때문에 PARTITION(A, p, r)이 작동하면서 pivot의 왼쪽에 모든 원소가 정렬될 것이다.\\
이는 0개의 원소를 가지는 subarray, n-1개의 원소를 가지는 subarray로 나뉘는 Worst Case의 경우와 유사하고 다음과 같은 점화식으로 표현될 수 있다.\\
$T(n) = T(n-1) + T(0) + \Theta(n)$\\
$T(n) = T(n-1) + \Theta(n)$\\
$T(n) = \Theta(n^2)$\\
결국 running time은 $\Theta(n^2)$가 될 것이다.\\
\\
\\
\subsection{ii) (5 point)}
\\
\begin{algorithmic}
 \STATE QUICKSORT\_NEW(A,p,r)
\end{algorithmic}
  
\begin{algorithmic}
 \IF {$p < r$}
 \STATE{(q, t) = PARTITION\_NEW(A, p, r)}
 \STATE {QUICKSORT\_NEW(A, p, q-1)}
 \STATE {QUICKSORT\_NEW(A, t+1, r)}
\end{algorithmic}
\\
\\
\subsection{iii) (7 point)}
\\
\begin{algorithmic}
 \STATE PARTITION\_NEW(A,p,r)
\end{algorithmic}
  
\begin{algorithmic}
 \STATE x = A[p]
 \STATE q = p
 \STATE t = p
 \FOR {j = p + 1 to r}
 \IF {A[j] $<$ x}
 \STATE t = t + 1
 \STATE a = A[j]
 \STATE A[j] = A[t]
 \STATE A[t] = A[q]
 \STATE A[q] = a
 \STATE q = q + 1
 \ELSIF {A[j] == x}
 \STATE t = t + 1
 \STATE exchange A[t] with A[j]
 \ENDFOR
 \RETURN (q, t)

\end{algorithmic}

\end{document}
