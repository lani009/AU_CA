\documentclass{article}
\usepackage[utf8]{inputenc}
\usepackage{amsmath}
\usepackage{kotex}
\usepackage{listings}   % 새로운 패키지 추가
\usepackage{hyperref}
\usepackage{graphicx}

\begin{document}

\section{Exercise 3}
\subsection{}
\begin{center}
    \includegraphics[scale=.06]{./img/five_queen_problem.eps}
\end{center}
문서 상에서 정보가 잘 표시되지 않아, 아래의 링크로 이미지 첨부해드립니다.
\url{https://drive.google.com/file/d/1cp6DiL01rb9r1U_OdtLQHuorETtQSUM_/view?usp=sharing}

\subsection{}
\begin{lstlisting}
    solution = list()

    def N_Queens_new(row, n, N, board):
        if n == 0:
            return True
    
        for j in range(1, N+1):
            if not is_attack(row, j, board, N):
                board[row][j] = 1
                solution.append((row, j))
                if N_Queens_new(row+1, n-1, N, board):
                    if n == N:
                        return solution
                    else:
                        return True
    
                board[row][j] = 0
                solution.remove((row, j))
    
        return False
    
    def is_attack(i, j, board, N):
        for k in range(1, i):
            if board[k][j]==1:
                return True
        k = i-1
        l = j-1
        while k >= 1 and l >= 1:
            if board[k][l]==1:
                return True
            k = k-1
            l = l-1
        k = i-1
        l = j+1
        while k >= 1 and l <= N:
            if board[k][l]==1:
                return True
            k = k-1
            l = l+1
    
        return False
\end{lstlisting}

\end{document}
