\documentclass{article}
\usepackage[utf8]{inputenc}
\usepackage{algorithm} 
\usepackage{algpseudocode}
\usepackage{amsmath}
\usepackage{kotex}

\begin{document}

\section{Exercise 3}
\subsection{}


\begin{algorithm}
	\caption{Recursive insertion sort}
	\begin{algorithmic}[1]
	    \Procedure{Insertion\_sort\_new}{$A, n$} \Comment {$A[1...n]$}
            \For {$j = 2$ to $A.length$}
                \State{$key \gets A[j]$}
                \State{$i \gets j -1$}
                \State{$index \gets$ binary\_search(A, 1, i, key)}
                
                \While{$i \geq  index$}
                    \State{$A[i+1] \gets A[i]$}
                    \State{$i \gets i - 1$}
                \EndWhile
                \State{$A[i+1] \gets key$}
            \EndFor
            \State{return $A$}
        \EndProcedure
	\end{algorithmic}
\end{algorithm}

\subsection{}
이진 탐색을 사용하면, $O(n \log n)$ 복잡도를 갖는다.
$j$번째 iteration에서 이진 탐색 procedure는 $log_2 (j-1)$ 만큼의 비교를 진행하기 때문에,\\

\begin{equation}
    \begin{split}
        \sum_{j=2}^{n} \log_{2} (j-1) & = \sum_{j=1}^{n-1} \log_{2} (j)\\
     & = n \log_{2} (n+1) + 2^{\log_{2} (n+1) + 1} + 2 \\
     & = O(n \log n)
    \end{split}
\end{equation}
으로 정리할 수 있다.

\subsection{}
$T(n) = c_1 n + c_2 (n-1) + c_3 (n-1) + c_4 \log(n-1) + \\
        c_5 \sum_{j=2}^{n} t_j + c_6 \sum_{j=2}^{n} (t_j-1) + c_7 \sum_{j=2}^{n} (t_j-1) + c_8 (n-1)$
        
에서 최악의 상황일 때 $t_j = j$가 된다. 따라서 \\
$\sum_{j=2}^{n} (j-1) n(n-1)/2$
이므로 수행시간은 $an^{2}+bn+c+d log n$로 나타낼 수 있다.
따라서 최악의 상황에서 수행시간은 $O(n^{2})$
\end{document}
