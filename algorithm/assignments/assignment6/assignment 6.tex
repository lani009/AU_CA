\documentclass{article}
\usepackage[utf8]{inputenc}
\usepackage{amsmath}
\usepackage{kotex}
\usepackage{listings}
\usepackage{hyperref}
\usepackage{graphicx}

\begin{document}

\section{Exercise 3}
프로그램 P가 입력값 i를 받았을 때 halt 하는지 알려주는 프로그램 Halt를 만들 수 있다고 가정한다.
프로그램 Halt는 P가 infinite loop에 빠졌을 경우 False를 반환하고, halt 할 경우 True를 반환한다.
이 때, P는 아래의 동작을 수행한다.

\subsection{}
\begin{lstlisting}
    def P(i):
        if Halt(P, i) == True:
            while True:
                do something...
        else:
            return 0
\end{lstlisting}

Halt(P, P)를 실행한다고 하자.
그렇다면, 이 프로그램은 P(P)가 halt할 경우 True를 반환할 것이고, P(P)가 infinite loop에 빠질 경우 False를 반환할 것이다.

\begin{enumerate}
    \item Halt(P, P)가 True를 반환할 경우

    Halt(P, P)가 True를 리턴하므로 P(P)가 halt해야 한다. 하지만 P(i) 프로그램의 if문에서 Halt(P, P)가 True조건에 해당하므로 infinite loop에 빠지게 된다.
    이럴 경우 P(P)는 halt하지 않고, infinite looping 하므로 Halt(P, P)의 반환값은 전제에 모순된다.

    \item Halt(P, P)가 False를 반환할 경우

    Halt(P, P)가 False를 리턴하므로 P(P)가 infinite looping 해야 한다. 하지만 P(i) 프로그램의 if문에서 Halt(P, P)가 False조건에 해당하므로 반환값 0과 함께 halt 하게 된다.
    이럴 경우 P(P)는 infinite looping하지 않고, halt 하므로 Halt(P, P)의 반환값은 전제에 모순된다.
\end{enumerate}

위 두가지 모두에서 전제가 모순되므로, 프로그램 P가 halt하는지 알려주는 프로그램 Halt를 만들 수 있다는 가정은 기각된다.


강의 내용에서는 Halt(P, P)가 True를 반환할 경우 Test가 infinite looping 한다고 가정하였습니다. 그 반대의 경우에는 Test가 halt 한다고 가정하였으나,
실제 Test 프로그램의 코드는 두 경우에서 모두 다항시간 내에 True 또는 False를 반환하도록 작성되어 있었습니다. 이 부분이 잘 이해가 되지 않아서, P(i) 프로그램이
while True를 통해서 infinite looping을 하도록 코드를 수정하였습니다. 반대의 경우에는 다항시간 내에 0을 반환하도록 하여, 제가 이해한 대로 문제를 풀이하였습니다.

\end{document}
